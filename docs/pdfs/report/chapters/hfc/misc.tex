\chapter{Misc stuff. To clean up.}
\label{hfc:misc}

\todo{This is just a collection of random snippets right now.}

\section{Ledger}

\subsection{Invalid states}
\label{hfc:ledger:invalid-states}

\todo{This came from the Byron/Shelley appendix. Need to generalise a bit or provide context.}
In a way, it is somewhat strange to have the hard fork mechanism be part of the
Byron (\cref{byron:hardfork}) or Shelley ledger (\cref{shelley:hardfork})
itself, rather than some overarching ledger on top. For Byron, a Byron ledger
state where the \emph{major} version is the (predetermined) moment of the hard
fork is basically an invalid state, used only once to translate to a Shelley
ledger. Similar, the \emph{hard fork} part of the Shelley protocol version will
never increase during Shelley's lifetime; the moment it \emph{does} increase,
that Shelley state will be translated to the (initial) state of the post-Shelley
ledger.

\section{Keeping track of time}
\label{hfc:time}

EpochInfo

\section{Failed attempts}

\subsection{Forecasting}
\label{hfc:failed:forecasting}

As part of the integration of any ledger in the consensus layer (not HFC
specific), we need a projection from the ledger \emph{state} to the consensus
protocol ledger \emph{view}
(\cref{,ledger:api:LedgerSupportsProtocol}).
As we have seen\todo{Once we write these sections, add back references here},
the HFC additionally requires for each pair of consecutive eras a  \emph{state}
translation functions as well as a \emph{projection} from the state of the old
era to the ledger view of the new era. These means that if we have $n + 1$ eras,
we need $n$ across-era projection functions, in addition to the $n + 1$
projections functions we already have \emph{within} each era.

This might feel a bit cumbersome; perhaps a more natural approach would be to
only have within-era projection functions, but require a function to translate
the ledger view (in addition to the ledger state) for each pair of eras.
We initially tried this approach; when projecting from an era to the next,
we would first ask the old era to give us the final ledger view in that era,
and then translate this final ledger view across the eras:

\begin{center}
\begin{tikzpicture}[
square/.style={rectangle, draw},
]
% old ledger
\node[square] (Astate) {old ledger state};
\node[square] (Aview1) [below=of Astate] {view};
\node[square] (Aview2) [right=of Aview1] {view};
\node         (Adots)  [right=of Aview2] {$\ldots$};
\node[square] (AviewN) [right=of Adots]  {view};
\draw[->] (Astate.south) -- (Aview1.north);
\draw[->] (Astate.south) .. controls +(down:1cm) and +(up:1cm).. (Aview2.north);
\draw[->] (Astate.south) .. controls +(down:1cm) and +(up:1cm).. (AviewN.north);
%
% some intermediate nodes for positiiong
\node (AstateN) [above=of AviewN] {};
\node (mid) [right=of AstateN] {};
\node (midH) [above=of mid] {era boundary};
\node (midM) [below=of mid] {};
\node (midL) [below=of midM] {};
%
% new ledger
\node[square] (Bstate) [right=of mid] {new ledger state};
\node[square] (Bview1) [below=of Bstate] {view};
\node[square] (Bview2) [right=of Bview1] {view};
\node         (Bdots)  [right=of Bview2] {$\ldots$};
\node[square] (BviewN) [right=of Bdots]  {view};
\draw[->] (Bstate.south) -- (Bview1.north);
\draw[->] (Bstate.south) .. controls +(down:1cm) and +(up:1cm).. (Bview2.north);
\draw[->] (Bstate.south) .. controls +(down:1cm) and +(up:1cm).. (BviewN.north);
%
\draw[dotted] (midH) -- (midL);
\draw[->, dashed] (AviewN.south) .. controls +(down:1cm) and +(down:1cm) .. (Bview2.south) node[pos=0.5, below] {\emph{translate}};;
\end{tikzpicture}
\end{center}

The problem with this approach is that the ledger view only contains a small
subset of the ledger state; the old ledger state might contain information about
scheduled changes that should be taken into account when constructing the ledger
view in the new era, but the final ledger view in the old era might not have
that information.

Indeed, a moment's reflection reveals that this cannot be right the approach.
After all, we cannot step the ledger state; the dashed arrow in
%
\begin{center}
\begin{tikzpicture}[
square/.style={rectangle, draw},
]
\node[square] (state) {ledger state at anchor};
\node[square] (view1) [below=of state] {view};
\node[square] (view2) [right=of view1] {view};
\node         (dots)  [right=of view2] {$\ldots$};
\node[square] (viewN) [right=of dots]  {view};
\draw[->] (state.south) -- (view1.north);
\draw[->] (state.south) .. controls +(down:1cm) and +(up:1cm).. (view2.north);
\draw[->] (state.south) .. controls +(down:1cm) and +(up:1cm).. (viewN.north);
\draw[->, dashed] (view1.south) .. controls +(down:1cm) and +(down:1cm) .. (view2.south) node[pos=0.5, below] {\emph{(impossible)}};
\end{tikzpicture}
\end{center}
%
is not definable: scheduled changes are recorded in the ledger state, not in
the ledger view. If we cannot even do this \emph{within} an era, there is no
reason to assume it would be possible \emph{across} eras.

We cannot forecast directly from the old ledger state to the new era either:
this would result in a ledger view from the old era in the new era, violating
the invariant we discussed in \cref{hfc:ledger:invalid-states}.

Both approaches---forecasting the final old ledger state and then translating,
or forecasting directly across the era boundary and then translating---also
suffer from another problem: neither approach would compute correct forecast
bounds. Correct bounds depend on properties of both the old and the new ledger,
as well as the distance of the old ledger state to that final ledger view. For
example, if that final ledger view is right at the edge of the forecast range of
the old ledger state, we should not be able to give a forecast in the new era at
all.

Requiring a special forecasting function for each pair of eras of course in a
way is cheating: it pushes the complexity of doing this forecasting to the
specific ledgers that the HFC is instantiated at. As it turns out, however, this
function tends to be easy to define for any pair of concrete ledgers; it's just
hard to define in a completely general way.
