\chapter{Mini protocol servers}
\label{servers}

The division of work between the network layer and the consensus layer when it
comes to the implementation of the clients and servers of the mini protocols is
somewhat pragmatic. Servers and clients that do significant amounts of network
layer logic (such as block fetch client which is making delta-Q related
decisions, node-to-node transaction server and client, which are dealing with
transaction windows, etc), live in the network layer. Clients and servers that
primarily deal with consensus side concerns live in the consensus layer; the
chain sync client (\cref{chainsyncclient}), is the primary example of this.
There are also a number of servers for the mini protocols that do little more
than provide glue code between the mini protocol and the consensus interface;
these servers are described in this chapter.

\section{Local state query}
\label{servers:lsq}

\section{Chain sync}
\label{servers:chainsync}

\section{Local transaction submission}
\label{servers:txsubmission}

Unlike remote (node to node) transaction submission, local (client to node)
transaction submission does not deal with transaction windows, and is
consequently much simpler; it therefore lives consensus side rather than
network side.

\section{Block fetch}
\label{servers:blockfetch}
