\chapter{Introduction}

The Cardano Consensus and Storage layer, or \emph{the consensus layer} for
short, is a critical piece of infrastructure in the Cardano Node. It
orchestrates between the \emph{network layer} below it and the
\emph{ledger layer} above it.

The network layer is a highly concurrent piece of software that deals with
low-level concerns; its main responsibility is to transmit data efficiently
across the network. Although it primarily transmits blocks and block headers, it
does not interpret them and does not need to know much about them. In the few
cases where it \emph{does} need to make some block-specific decisions, it
calls back into the consensus layer to do so.

The ledger layer by contrast exclusively deals with high-level concerns. It is
entirely stateless: its main responsibility is to define a single pure
function describing how the ledger state is transformed by blocks (verifying
that blocks are valid in the process). It is only concerned with linear history;
it is not aware of the presence of multiple competing chains or the roll backs
required when switching from one chain to another. We do require that the ledger
layer provides limited \emph{lookahead}, computing (views on near)
\emph{future} ledger states (required to be able to validate block headers
without access to the corresponding block bodies)

The consensus layer mediates between these two layers. It includes a
bespoke storage layer that provides efficient access to the current ledger state
as well as recent \emph{past} ledger states (required in order to be able
to validate and switch to competing chains). The storage layer also
provides direct access to the blocks on the blockchain itself, so that they can
be efficiently streamed to clients (via the network layer). When there are
competing chains, the consensus layer decides which chain is preferable and
should be adopted, and it decides when to \emph{contribute} to the chain
(produce new blocks). All ``executive decisions'' about the chain are made in
and by the consensus layer.

Lastly, as well we see, the consensus layer is highly abstract and places a
strong emphasis on compositionality, making it usable with many different
consensus algorithms and ledgers. Importantly, compositionality enables the
\emph{hard fork combinator} to combine multiple ledgers and regard them as a
single blockchain.

The goal of this document is to outline the design goals for the consensus
layer, how we achieved them, and where there is still scope for improvement. We
will both describe \emph{what} the consensus layer is, and \emph{why} it is the
way it is. Throughout we will also discuss what \emph{didn't} work, approaches
we considered but rejected, or indeed adopted but later abandoned; discussing
these dead ends is sometimes at least as informative as discussing the solution
that did work.

We will consider some of the trade-offs we have had to make, how they
affected the development, and discuss which of these trade-offs should perhaps
be reconsidered. We will also take a look at how the design can scale to
facilitate future requirements, and which requirements will be more problematic
and require more large-scale refactoring.

The target audience for this document is primarily developers working on the
consensus layer. It may also be of more broader interest to people generally
interested in the Cardano blockchain, although we will assume that the
reader has a technical background.
