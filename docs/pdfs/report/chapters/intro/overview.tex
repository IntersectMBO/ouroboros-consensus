\chapter{Overview}

\section{Components}

\subsection{Consensus protocols}
\label{overview:consensus}

The consensus protocol has two primary responsibilities:
\label{consensus-responsibilities}

\begin{description}
\item[Chain selection] Competing chains arise when two or more nodes extend the
chain with different blocks. This can happen when nodes are not aware of each
other's blocks due to temporarily network delays or partitioning, but depending
on the particular choice of consensus algorithm it can also happen in the normal
course of events. When it happens, it is the responsibility of the consensus
protocol to choose between these competing chains.

\item[Leadership check] In proof-of-work blockchains any node can produce a
block at any time, provided that they have sufficient hashing power. By
contrast, in proof-of-stake time is divided into \emph{slots}, and each slot has
a number of designated \emph{slot leaders} who can produce blocks in that slot.
It is the responsibility of the consensus protocol to decide on this mapping
from slots to slot leaders.
\end{description}

The consensus protocol will also need to maintain its own state; we will discuss
state management in more detail in \cref{storage:inmemory}.

\subsection{Ledger}
\label{overview:ledger}

The role of the ledger is to define what is stored \emph{on} the blockchain.
From the perspective of the consensus layer, the ledger has three primary
responsibilities:

\begin{description}
\item[Applying blocks] The most obvious and most important responsibility of
the ledger is to define how the ledger state changes in response to new blocks,
validating blocks at it goes and rejecting invalid blocks.\

\item[Applying transactions] Similar to applying blocks, the ledger layer also
must provide an interface for applying a single transaction to the ledger state.
This is important, because the consensus layer does not just deal with
previously constructed blocks, but also constructs \emph{new} blocks.

\item[Ticking time] Some parts of the ledger state change due to the passage of
time only. For example, blocks might \emph{schedule} some changes to be applied
later, and then when the relevant slot arrives those changes should be applied,
independent from any blocks.

\item[Forecasting] Some consensus protocols require limited information from the
ledger. In Praos, for example, a node's probability of being a slot leader is
proportional to its stake, but the stake distribution is something that the
ledger keeps track of. We refer to this as a \emph{view} on the ledger, and we
require not just that the ledger can give us a view on the \emph{current} ledger
state, but also \emph{predict} what that view will be for slots in the near
future. We will discuss the motivation for this requirement in
\cref{nonfunctional:network:headerbody}.
\end{description}

The primary reason for separating out ``ticking'' from applying blocks is that
the consensus layer is responsible to the leadership check
(\cref{consensus-responsibilities}), and when we need to decide if we should be
producing a block in a particular slot, we need to know the ledger state at that
slot (even though we don't have a block for that slot \emph{yet}). It is also
required in the mempool; see \cref{mempool}.

\section{Design Goals}

\subsection{Multiple consensus protocols}
\label{multiple-consensus-protocols}

From the beginning it was clear that we would need support for multiple
consensus algorithms: the Byron era uses a consensus algorithm called
(Permissive) BFT (\cref{bft}) and the Shelley era uses a consensus algorithm
called Praos (\cref{praos}). Moreover, the Cardano blockchain is a \emph{hybrid}
chain where the prefix of the chain runs Byron (and thus uses BFT), and then
continues with Shelley (and thus uses Praos); we will come back to the topic of
composing protocols when we discuss the hard fork combinator (\cref{hfc}). It is
therefore important that the consensus layer abstracts over a choice of
consensus protocol.

\subsection{Support for multiple ledgers}
\label{multiple-ledgers}

For much the same reason that we must support multiple consensus protocols, we
also have to support multiple ledgers. Indeed, we expect more changes in ledger
than in consensus protocol; currently the Cardano blockchain starts with a
Byron ledger and then transitions to a Shelley ledger, but further changes to
the ledger have already been planned (some intermediate ledgers currently
code-named Allegra and Mary, as well as larger updates to Goguen, Basho and
Voltaire). All of the ledgers (Shelley up to including Voltaire)
use the Praos consensus algorithm (potentially extended with the genesis chain
selection rule, see \cref{genesis}).

\subsection{Decouple consensus protocol from ledger}
\label{decouple-consensus-ledger}

As we saw above (\cref{multiple-ledgers}), we have multiple ledgers that all
use the same consensus protocol. We therefore should be able to define the
consensus protocol \emph{independent} from a particular choice of ledger,
merely defining what the consensus protocol expects from the ledger
(we will see what this interface looks like in \cref{ledger}).

\subsection{Testability}
\label{testability}

The consensus layer is a critical component of the Cardano Node, the software
that runs the Cardano blockchain. Since the blockchain is used to run the ada
cryptocurrency, it is of the utmost importance that this node is reliable;
network downtime or, worse, corruption of the blockchain, cannot be tolerated.
As such the consensus layer is subject to much stricter correctness criteria
than most software, and must be tested thoroughly. To make this possible, we
have to design for testability.

\begin{itemize}
\item We must be able to simulate various kinds of failures (disk
failures, network failures, etc.) and verify that the system can recover.
\item We must be able to run \emph{lots} of tests which means that tests need to
be cheap. This in turn will require for example the possibility to swap the
cryptographic algorithms for much faster ``mock'' crypto algorithms.
\item We must be able to test how the system behaves under certain
expected-but-rare circumstances. For example, under the Praos consensus
protocol it can happen that a particular slot has multiple leaders. We should be
able to test what happens when this happens repeatedly, but the leader selection
is a probabilistic process; it would be difficult to set up test scenarios to
test for this specifically, and even more difficult to set things up so that
those scenarios are \emph{shrinkable} (leading to minimal test cases). We must
therefore be able to ``override'' the behaviour of the consensus protocol (or
indeed the ledger) at specific points.
\item We must be able to test components individually (rather than just the
system as a whole), so that if a test fails, it is much easier to see where
something went wrong.
\end{itemize}

\subsection{Adaptability and Maintainability}
\label{adaptability}

The Cardano Node began its life as an ambitious replacement of the initial
implementation of the Cardano blockchain, which had been developed by Serokell.
At the time, the Shelley ledger was no more than a on-paper design, and
the Praos consensus protocol existed only as a research paper. Moreover, since
the redesign would be unable to reuse any parts of the initial implementation,
even the Byron ledger did not yet exist when the consensus layer was started.
It was therefore important from the get-go that the consensus layer was not
written for a specific ledger, but rather abstract over a choice of ledger
and define precisely what the responsibilities of that ledger were.

This abstraction over both the consensus algorithm and the ledger is important
for other reasons, too. As we've mentioned, although initially developed to
support the Byron ledger and the (Permissive) BFT consensus algorithm, the goal
was to move to Shelley/Praos as quickly as possible. Moreover, additional
ledgers had already been planned (Goguen, Basho and Voltaire), and research on
consensus protocols was (and still is) ongoing. It was therefore important that
the consensus layer could easily be adapted.

Admittedly, adaptability does not \emph{necessarily} require abstraction. We
could have built the consensus layer against the Byron ledger initially
(although we might have had to wait for it to be partially completed at least),
and then generalise as we went. There are however a number of downsides to this
approach.

\begin{itemize}
\item When working with a concrete interface, it is difficult to avoid certain
assumptions creeping in that may hold for this ledger but will not hold for
other ledgers necessarily. When such assumptions go unnoticed, it can be costly
to adjust later. (For one example of such an assumption that nonetheless
\emph{did} go unnoticed, despite best efforts, and took a lot of work to
resolve, see \cref{time} on removing the assumption that we can always
convert between wallclock time and slot number.)

\item IOHK is involved in the development of blockchains other than the public
Cardano instance, and from the start of the project, the hope was that the
consensus layer can be used in those projects as well. Indeed, it is currently
being integrated into various other IOHK projects.

\item Perhaps most importantly, if the consensus layer only supports a single,
concrete ledger, it would be impossible to \emph{test} the consensus layer with
any ledgers other than that concrete ledger. But this means that all consensus
tests need to deal with all the complexities of the real ledger. By contrast,
by staying abstract, we can run a lot of consensus tests with mock ledgers that
are easier to set up, easier to reason about, more easily instrumented and more
amenable to artificially creating rare circumstances (see \cref{testability}).
\end{itemize}

Of course, abstraction is also just good engineering practice. Programming
against an abstract interface means we are clear about our assumptions,
decreases dependence between components, and makes it easier to understand and
work with individual components without having to necessarily understand the
entire system as a whole.

\subsection{Composability}
\label{composability}

The consensus layer is a complex piece of software;  at the time we are writing
this technical report, it consists of roughly 100,000 lines of code. It is
therefore important that we split it into into small components that can be
understood and modified independently from the rest of the system. Abstraction,
discussed in \cref{adaptability}, is one technique to do that, but by no means
the only. One other technique that we make heavy use of is composability. We
will list two examples here:

\begin{itemize}
\item As discussed in \cref{multiple-consensus-protocols} and
\cref{multiple-ledgers}, the Cardano blockchain has a prefix that runs the BFT
consensus protocol and the Byron ledger, and then continues with the Praos
consensus protocol and the Shelley ledger. We do not however define a consensus
protocol that is the combination of Byron and Praos, nor a ledger that is the
combination of Byron and Shelley. Instead, the \emph{hard fork combinator}
(\cref{hfc}) makes it possible to \emph{compose} consensus protocols and
ledgers: construct the hybrid consensus protocol from an implementation of BFT
and an implementation of Praos, and similarly for the ledger.

\item We mentioned in \cref{testability} that it is important that we can
test the behaviour of the consensus layer under rare-but-possible circumstances,
and that it is therefore important that we can override the behaviour of the
consensus algorithm in tests. We do not accomplish this however by adding
special hooks to the Praos consensus algorithm (or any other); instead we define
another combinator that takes the implementation of a consensus algorithm and
\emph{adds} additional hooks for the sake of the testing infrastructure. This
means that the implementation of Praos does not have to be aware of testing
constraints, and the combinator that adds these testing hooks does not need to
be aware of the details of how Praos is implemented.
\end{itemize}

\subsection{Predictable Performance}

Make sure node operators do not set up nodes for "normal circumstances" only
for the network to go down when something infrequent (but expected) occurs.
(This is not about malicious behaviour, that's the next section).

\duncan

\subsection{Protection against DoS attacks}

Brief introduction to asymptotic attacker/defender costs. (This is just an
overview, we come back to these topics in more detail later.)

\duncan

\section{History}
\label{overview:history} % OBFT references refer to this section as well

\duncan

\begin{itemize}
\item Briefly describe the old system (\lstinline!cardano-sl!) the decision
to rewrite it
\item Briefly discuss the OBFT hard fork.
\end{itemize}
