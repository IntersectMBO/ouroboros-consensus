\newcommand{\Val}{\fun{Val}}
\newcommand{\POV}[1]{\ensuremath{\mathsf{PresOfVal}(\mathsf{#1})}}
\newcommand{\DBE}[2]{\ensuremath{\mathsf{DBE}({#1},~{#2})}}
\newcommand{\DGO}[2]{\ensuremath{\mathsf{DGO}({#1},~{#2})}}
\newcommand{\transtar}[2]{\xlongrightarrow[\textsc{#1}]{#2}\negthickspace^{*}}

\section{Properties}
\label{sec:properties}

This section describes the properties that the ledger should have. The goal is to
to include these properties in the executable specification to enable e.g.
property-based testing or formal verification.

% TODO - the hand proofs of preservation of ada and non-negative deposit
% pot are woefully out of date, mostly due to the removal of decaying deposits.
%\input{hand_proofs}

\subsection{Header-Only Validation}
\label{sec:header-only-validation}
The header-only validation properties of the Shelley Ledger are the analogs
of those from Section 8.1 of \cite{byron_chain_spec}.

In any given chain state, the consensus layer needs to be able to validate the
block headers without having to download the block bodies.
Property~\ref{prop:header-only-validation} states that if an extension of a
chain that spans less than $\StabilityWindow$ slots is valid, then validating the
headers of that extension is also valid. This property is useful for its
converse: if the header validation check for a sequence of headers does not
pass, then we know that the block validation that corresponds to those headers
will not pass either.

First we define the header-only version of the $\mathsf{CHAIN}$ transition,
which we call $\mathsf{CHAINHEAD}$.
It is very similar to $\mathsf{CHAIN}$, the differences being:
\begin{itemize}
  \item The $\mathsf{CHAINHEAD}$ signal is not a block, but a block header ($\BHeader$).
  \item $\mathsf{CHAINHEAD}$ does not call $\mathsf{BBODY}$.
  \item $\mathsf{CHAINHEAD}$ does not call $\mathsf{TICK}$, but instead
    calls the similar $\mathsf{TICKF}$, which differs by
    not calling the reward update transition $\mathsf{RUPD}$.
  \item $\mathsf{CHAINHEAD}$ does not store the new epoch state $\var{nes}$ in
    its state, but rather contains it in the environment.
    We will conveniently \textbf{abuse the tuple notation} and write
    $(nes, \tilde{s}) = s$
    for splitting the chain state into the new epoch state and the remaiing fields.
\end{itemize}

\begin{figure}[ht]
  \begin{equation}\label{eq:tickf}
    \inference[TickForecast]
    {
      {
        \vdash
        \var{nes}
        \trans{newepoch}{\epoch{slot}}
        \var{nes}'
      }
      \\~\\
      (\var{e_\ell'},~\var{b_{prev}'},~\var{b_{cur}'},~\var{es'},~\var{ru'},~\var{pd'})
      \leteq\var{nes'}
      \\
      \var{es''}\leteq\fun{adoptGenesisDelegs}~\var{es'}~\var{slot}
      \\
      \var{forecast}\leteq
      (\var{e_\ell'},~\var{b_{prev}'},~\var{b_{cur}'},~\var{es''},~\var{ru'},~\var{pd'})
      \\~\\
    }
    {
      \vdash\var{nes}\trans{tickf}{\var{slot}}\varUpdate{\var{forecast}}
    }
  \end{equation}
  \caption{Tick Forecast rules}
  \label{fig:rules:tickf}
\end{figure}

\clearpage

\begin{figure}[ht]
  \begin{equation}\label{eq:chain-head}
    \inference[ChainHead]
    {
      \var{bhb} \leteq \bhbody{bh}
      &
      \var{s} \leteq \bslot{bhb}
      \\~\\
      \fun{prtlSeqChecks}~\var{lab}~\var{bh}
      \\~\\
      {
        \vdash\var{nes}\trans{\hyperref[fig:rules:tickf]{tickf}}{\var{s}}\var{forecast}
      } \\~\\
      (\var{e_1},~\wcard,~\wcard,~\wcard,~\wcard,~\wcard)
        \leteq\var{nes} \\
      (\var{e_2},~\wcard,~\wcard,~\var{es},~\wcard,~\wcard,~\var{pd})
        \leteq\var{forecast} \\
        (\var{acnt},~\wcard,\var{ls},~\wcard,~\var{pp})\leteq\var{es}\\
          \var{ne} \leteq  \var{e_1} \neq \var{e_2}\\
          \eta_{ph} \leteq \prevHashToNonce{(\lastAppliedHash{lab})} \\~\\
      \fun{chainChecks}~
        \MaxMajorPV~(\fun{maxHeaderSize}~\var{pp},~\fun{maxBlockSize}~\var{pp},~\fun{pv}~\var{pp})~
        \var{bh}\\~\\
      {
        {\begin{array}{c}
        \var{pp} \\
        \eta_c \\
        \eta_\var{ph} \\
        \end{array}}
        \vdash
        {\left(\begin{array}{c}
        \eta_0 \\
        \eta_h \\
        \end{array}\right)}
        \trans{\hyperref[fig:rules:tick-nonce]{tickn}}{\var{ne}}
        {\left(\begin{array}{c}
        \eta_0' \\
        \eta_h' \\
        \end{array}\right)}
      }\\~\\~\\
      {
        {\begin{array}{c}
            \var{pd} \\
            \eta_0' \\
         \end{array}}
        \vdash
        {\left(\begin{array}{c}
              \var{cs} \\
              \eta_v \\
              \eta_c \\
        \end{array}\right)}
        \trans{\hyperref[fig:rules:prtcl]{prtcl}}{\var{bh}}
        {\left(\begin{array}{c}
              \var{cs'} \\
              \eta_v' \\
              \eta_c' \\
        \end{array}\right)}
      } \\~\\~\\
      \var{lab'}\leteq (\bblockno{bhb},~\var{s},~\bhash{bh} ) \\
    }
    {
      \var{nes}
      \vdash
      {\left(\begin{array}{c}
            \var{cs} \\
            \eta_0 \\
            \eta_v \\
            \eta_c \\
            \eta_h \\
            \var{lab} \\
      \end{array}\right)}
      \trans{chainhead}{\var{bh}}
      {\left(\begin{array}{c}
            \var{cs}' \\
            \eta_0' \\
            \eta_v' \\
            \eta_c' \\
            \eta_h' \\
            \var{lab}' \\
      \end{array}\right)}
    }
  \end{equation}
  \caption{Chain-Head rules}
  \label{fig:rules:chainhead}
\end{figure}

\begin{property}[Header only validation]\label{prop:header-only-validation}
  For all states $s$ with slot number $t$\footnote{i.e. the
    component $\var{s_\ell}$ of the last applied block of $s$ equals $t$},
    and chain extensions $E$ with corresponding headers $H$ such that:
  %
  $$
  0 \leq t_E - t  \leq \StabilityWindow
  $$
  %
  we have:
  %
  $$
  \vdash s \transtar{\hyperref[fig:rules:chain]{chain}}{E} s'
  \implies
  \var{nes} \vdash \tilde{s} \transtar{\hyperref[fig:rules:chainhead]{chainhead}}{H} s''
  $$
  where $s=(\var{nes},~\tilde{s})$,
  $t_E$ is the maximum slot number appearing in the blocks contained in
  $E$, and $H$ is obtained from $E$ by applying $\fun{bheader}$ to each block in $E$.
\end{property}

\begin{property}[Body only validation]\label{prop:body-only-validation}
  For all states $s$ with slot number $t$, and chain
  extensions $E = [b_0, \ldots, b_n]$ with corresponding headers $H$ such that:
  $$
  0 \leq t_E - t  \leq \StabilityWindow
  $$
  we have that for all $i \in [1, n]$:
  $$
  \var{nes} \vdash \tilde{s} \transtar{\hyperref[fig:rules:chainhead]{chainhead}}{H} s_{h}
  \wedge
  \vdash (\var{nes},~\tilde{s}) \transtar{\hyperref[fig:rules:chain]{chain}}{[b_0 \ldots b_{i-1}]} s_{i-1}
  \implies
  \var{nes'} \vdash \tilde{s}_{i-1}\trans{\hyperref[fig:rules:chainhead]{chainhead}}{h_i} s'_{h}
  $$
  where $s_{i-1}=(\var{nes'},~\tilde{s}_{i-1})$,
  $t_E$ is the maximum slot number appearing in the blocks contained in $E$.
\end{property}

Property~\ref{prop:body-only-validation} states that if we validate a sequence
of headers, we can validate their bodies independently and be sure that the
blocks will pass the chain validation rule. To see this, given an environment
$e$ and initial state $s$, assume that a sequence of headers
$H = [h_0, \ldots, h_n]$ corresponding to blocks in $E = [b_0, \ldots, b_n]$ is
valid according to the $\mathsf{chainhead}$ transition system:
%
$$
\var{nes} \vdash \tilde{s} \transtar{\hyperref[fig:rules:chainhead]{chainhead}}{H} \tilde{s'}
$$
%
Assume the bodies of $E$ are valid
according to the $\mathsf{bbody}$ rules, but $E$ is not valid according to
the $\mathsf{chain}$ rule. Assume that there is a $b_j \in E$ such that it is
\textbf{the first block} such that does not pass the $\mathsf{chain}$
validation. Then:
%
$$
\vdash (\var{nes},~\tilde{s}) \transtar{\hyperref[fig:rules:chain]{chain}}{[b_0, \ldots b_{j-1}]} s_j
$$
But by Property~\ref{prop:body-only-validation} we know that
%
$$
\var{nes}_j \vdash \tilde{s}_j \trans{\hyperref[fig:rules:chainhead]{chainhead}}{h_j} \tilde{s}_{j+1}
$$
which means that block $b_j$ has valid headers, and this in turn means that the
validation of $b_j$ according to the chain rules must have failed because it
contained an invalid block body. But this contradicts our assumption that the
block bodies were valid.

\clearpage

%%% Local Variables:
%%% mode: latex
%%% TeX-master: "ledger-spec"
%%% End:
