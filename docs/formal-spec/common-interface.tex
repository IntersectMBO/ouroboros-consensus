\section{Common Interface}
\label{sec:common-interface}

This section defines the interface used by this specification, i.e.
the transition systems that are assumed to exist.

\subsection{Chain Tick Transition}
\label{sec:tick-trans}

The Chain Tick Transition performs some chain level upkeep.
\begin{figure}[htb]
  \emph{Chain Tick Transitions}
  \begin{equation*}
    \vdash \var{\_} \trans{tick}{\_} \var{\_} \subseteq
    \powerset (\NewEpochState \times \Slot \times \NewEpochState)
  \end{equation*}
  \caption{Tick transition-system types}
  \label{fig:ts-types:tick}
  %
\end{figure}

The type of the $\mathsf{TICK}$ transition is shown in Figure~\ref{fig:ts-types:tick}.

\subsection{Ledger Sequence Transition}
\label{sec:ledgers-trans}

The Ledger Sequence Transition processes a list of transactions.
\begin{figure}[htb]
  \emph{Ledger Sequence transitions}
  \begin{equation*}
    \_ \vdash
    \var{\_} \trans{ledgers}{\_} \var{\_}
    \subseteq \powerset ((\Slot\times\PParams\times\Coin) \times \LState \times \seqof{\Tx} \times \LState)
  \end{equation*}
  \caption{Ledger Sequence transition-system types}
  \label{fig:ts-types:ledgers}
\end{figure}

The type of the $\mathsf{LEDGERS}$ transition system is shown in Figure~\ref{fig:ts-types:ledgers}.

%%% Local Variables:
%%% mode: latex
%%% TeX-master: "ledger-spec"
%%% End:
